 %parameterizatio

\section{Policy Mirror Descent}
\label{subsec:revisitTRPO}

\newcommand{\real}{\mathbb{R}}


We begin with the development of the basic Policy Mirror Descent (PMD) strategy,
which will form the basis for our subsequent algorithms and their analyses.
As mentioned in the introduction, our analysis of this and subsequent methods
focuses on the \emph{non-convex} setting.

Consider the following local optimization problem:
given a \emph{reference policy} $\refPi$ (usually the current policy), 
maximize the proximal regularized expected reward,
using relative entropy as the regularizer:
%
\begin{equation}
\label{eq:max_expected_reward_plus_relative_entropy}
\pi_{\theta} = \argmax\limits_{\pi_\theta \in \Pi} { \ep\limits_{\rho \sim \pi_\theta}{  r(\rho)  - \tau \KL(\pi_\theta \| \refPi) } }.
\end{equation}
%
Relative entropy regularization has been widely investigated in
online learning and constrained optimization
\citep{nemirovskii1983problem,beck2003mirror},
primarily as a component of the mirror descent algorithm.
Observe that when $\refPi$ is the uniform distribution,
\cref{eq:max_expected_reward_plus_relative_entropy}
reduces to entropy regularized expected reward.
It is important to note that,
since we are interested in the non-convex setting
and only assume that
$\pi$ is parametrized as a smooth
function of $\theta \in \real^d$,
$\Pi$ is generally a non-convex subset of the simplex.
Therefore, 
\cref{eq:max_expected_reward_plus_relative_entropy}
is a difficult constrained optimization problem.

A useful way to decompose this constrained optimization
is to consider an alternating lift-and-project procedure
that isolates the different computational challenges.
%
\begin{equation}
\label{eq:pmd}
\begin{split}
\text{\bf (Project Step)} \quad &\argmin\limits_{\pi_\theta \in \Pi}{\KL( \pi_\theta \| \bar{\pi}_{\tau}^* )}, \\
\text{\bf (Lift Step)}  \quad & \text{where}\ \ \bar{\pi}_{\tau}^* =  \argmax\limits_{\pi \in \Delta}{ \ep\limits_{\rho \sim \pi}{  r(\rho)  - \tau \KL(\pi \| \refPi) } }. 
\end{split}
\end{equation}
%
Crucially, the reformulation \cref{eq:pmd} remains equivalent to the
original problem \cref{eq:max_expected_reward_plus_relative_entropy},
in that it preserves the same set of solutions,
as established in \cref{prop:mirrordescent_projection}.

\begin{prop}
\label{prop:mirrordescent_projection}
Given a \emph{reference policy} $\refPi$,
\[
 \argmax\limits_{\pi_\theta \in \Pi} { \ep\limits_{\rho \sim \pi_\theta}{  r(\rho)  - \tau \KL(\pi_\theta \| \refPi) } } 
 = \argmin\limits_{\pi_\theta \in \Pi}{ \KL(\pi_\theta \| \bar{\pi}_\tau^*) }.
\]
\end{prop}

Note that this result holds even for the non-convex setting.
%
The reformulation \cref{eq:pmd} immediately leads to our PMD algorithm:
Lift the current policy $\pi_{\theta_t}$ to $\bar{\pi}_\tau^*$,
then perform multiple steps of gradient descent in the Project Step
to update $\pi_{\theta_{t+1}}$.%
%
\footnote{
To estimate this gradient one would need to use self-normalized importance
sampling \cite{owen2013monte}.
We omit the details here since PMD is not our main algorithm;
similar techniques can be found in the implementation of REPMD. 
% in cref-tbd.
}
%Note that vanilla gradient descent methods for TRPO can be interpreted as performing only one step gradient descent for the project step. \todor[]{Is it correct?} 

When $\Pi$ is a convex set, one can show that PMD asymptotically converges to the optimal policy \citep{nemirovskii1983problem,beck2003mirror}. 
The next proposition shows that despite the non-convexity of $\Pi$,
PMD still enjoys desirable properties.

\begin{prop}
\label{prop:monoto_policymirrordescent}
Let $\pi_{\theta_{t}}$ denote the policy produced at step $t$ of
the update sequence.
Then PMD satisfies the following properties for an arbitrary 
parametrization of $\pi$.
\begin{enumerate}
	\item {\bf (Monotonic Improvement Guarantee)} 
	%The sequence of policies learned by TRPO is guaranteed to be monotonically improved:
	If the Project Step $\min\limits_{\pi_\theta \in \Pi}{\KL( \pi_\theta \| \bar{\pi}_{\tau}^* )}$ can be solved globally optimally, then
	 \begin{equation*}
	\oep_{\rho \sim \pi_{\theta_{t+1}}}{r(\rho)} - \oep_{\rho \sim \pi_{\theta_{t}}}{  r(\rho)} \ge 0.
	\end{equation*}
	\item  {\bf (Global Optimum Guarantee)} If the Project Step $\min\limits_{\pi_\theta \in \Pi}{\KL( \pi_\theta \| \bar{\pi}_{\tau}^* )}$ can be solved globally optimally, then $ \pi_{\theta_{t}}$ converges to the global optimum $\pi^*$, i.e. 
	\[
\lim_{t\rightarrow\infty}\oep_{\rho \sim \pi_{\theta_t}}{  r(\rho)}
=
	 \oep_{\rho \sim \pi^*}{  r(\rho)} \ge \oep_{\rho \sim \pi}{  r(\rho)}, \quad \forall \pi \in \Pi.
	\]
	\item {\bf (Fixed Points)} Assume that the Project Step is optimized by gradient descent, then the fixed points of PMD are the 
	 stationary points of the expected reward $\oep_{\rho \sim \pi_\theta}{  r(\rho)}$. 
\end{enumerate}
\end{prop}

Despite these desirable properties, 
\cref{prop:monoto_policymirrordescent}
relies on the condition that the Project Step in PMD is solved to
global optimality.
It is usually not practical to achieve such a stringent requirement
when $\pi_\theta$ is not convex in $\theta$,
limiting the applicability of \cref{prop:monoto_policymirrordescent}.

Another shortcoming with this naive strategy is 
that PMD typically gets trapped in poor local optima.
Indeed, while the relative entropy regularizer help prevent
a large policy update, it also tends to limit exploration.
Moreover, minimizing the KL divergence
$\KL(\pi_\theta \| \bar{\pi}_\tau^*)$
is known to be \emph{mode seeking} \citep{kevin2012machine},
which can lead to mode collapse during the learning process.
Once a policy $\bar{\pi}_\tau^*$ has lost important modes,
learning can easily become trapped at a sub-optimal policy.
Unfortunately, at such points,
the relative entropy regularizer does not encourage further exploration.

