


%\subsection{Cooperate with Value Function Approximation}
%\section{Combining REPMD with Value Function Approximation}
\section{An Actor-Critic Extension}
\label{subsec:repmd_value}

\newcommand{\parV}{V_{\phi}}
\newcommand{\parTargetV}{V_{\bar{\phi}}}
\newcommand{\parQ}{Q_{\psi}}
\newcommand{\parPi}{\pi_{\theta}}
\newcommand{\parQone}{Q_{\psi_1}}
\newcommand{\parQtwo}{Q_{\psi_2}}


Finally, we develop a natural extension of REPMD to an actor-critic
formulation by incorporating a value function approximator.
We refer to this final algorithm as Policy Mirror Actor-Critic (PMAC). 

It is well known that data efficiency of policy-based methods 
can be generally improved by adding a value-based critic.
For a reference policy $\refPi$ and an initial state $s$,
recall that the objective in the Lift Step of REPMD is 
\[
\gO_{\text{REPMD}}(\pi, s) =   \ep\limits_{\rho \sim \pi}{  r(\rho)  - \tau \KL(\pi \| \refPi) + \tau^{\prime} \cH(\pi)}, 
\]
where $\rho=  (s_1 = s, a_1, s_2, a_2, \ldots)$.
To incorporate value function approximation, we need to derive 
the temporal consistency structure for this objective, 
which can be done as follows. 
First, write
 \[
 \gO_{\text{REPMD}}(\pi, s) = \E_{a\sim \pi(\cdot \vert s)} \left[ r(s,a) + \gO_{\text{REPMD}}(\pi, s')  + \tau \log \refPi(a|s) - \left(\tau+ \tau' \right) \log \pi(a|s) \right]. 
 \]
Let $\bar{\pi}_{\tau,\tau^{\prime}}^* (\cdot|s) = \argmax_{\pi} \gO_{\text{REPMD}}(\pi, s) $ denote the optimal policy on state $s$. 
%
Further denote the soft optimal state-value function $\gO_{\text{REPMD}}(\bar{\pi}_{\tau,\tau^{\prime}}^* (\cdot|s), s)$ by $\bar{V}_{\tau,\tau^{\prime}}^*(s)$, and let  $\bar{Q}_{\tau,\tau^{\prime}}^*(s,a) = r(s,a) + \gamma \bar{V}_{\tau,\tau^{\prime}}^*(s')$ be the soft-Q function.
It can then be verified that 
%
\begin{equation}
\begin{split}
& \bar{V}_{\tau,\tau^{\prime}}^*(s) = (\tau + \tau') \log \sum_a \exp \left\{ \frac{\bar{Q}_{\tau,\tau^{\prime}}^*(s,a) + \tau \log \bar{\pi}(a|s)} {\tau + \tau'} \right\}; \\
& \bar{\pi}_{\tau,\tau^{\prime}}^* (a|s) = \exp \left\{ \frac{\bar{Q}_{\tau,\tau^{\prime}}^*(s,a) + \tau \log \bar{\pi}(a|s) - \bar{V}_{\tau,\tau^{\prime}}^*(s)}{\tau + \tau'} \right\}.
\end{split}
\label{soft-v-and-pi}
\end{equation}
 
We propose to train a
soft state-value function $\parV$ parameterized by $\phi$,
a soft Q-function $\parQ$ parameterized by $\psi$,
and a policy $\parPi$ parameterized by $\theta$,
based on \cref{eq:repmd}.
The update rules for these parameters can be derived as follows.

The soft state-value function approximates the soft optimal state-value $\bar{V}_{\tau,\tau^{\prime}}^*$. Note that we can re-express $\bar{V}_{\tau,\tau^{\prime}}^*$ by 
\begin{align*}
\bar{V}_{\tau,\tau^{\prime}}^*(s) 
%= & (\tau + \tau') \log \sum_a \refPi(a|s) \exp \left\{ \frac{\bar{Q}_{\tau,\tau^{\prime}}^*(s,a) - \tau' \log \bar{\pi}(a|s)} {\tau + \tau'} \right\} \\ 
= & (\tau + \tau') \log \E _ {a\sim \refPi} \left[ \exp \left\{ \frac{\bar{Q}_{\tau,\tau^{\prime}}^*(s,a) - \tau' \log \bar{\pi}(a|s)} {\tau + \tau'} \right\} \right].
\end{align*}
This suggests a Monte-Carlo estimate for
$\bar{V}_{\tau,\tau^{\prime}}^*(s)$:
by sampling one single action $a$ according to the reference policy $\refPi$,
we have $\bar{V}_{\tau,\tau^{\prime}}^*(s)
 \approx  \bar{Q}_{\tau,\tau^{\prime}}^*(s,a) - \tau' \log \bar{\pi}(a|s) $.
Then,
given a replay buffer $\gD$,
the soft state-value function can be trained to minimize
the mean squared error,
%
\begin{equation}
\label{eq:trainV}
L (\phi) = \E_{s\sim \gD} \left[ \frac{1}{2} \left( \parV(s) -  \left[ \parQ(s,a ) - \tau' \log \bar{\pi}(a|s) \right] \right)^2 \right]
.
\end{equation}

One might note that, in principle, there is no need to include a 
separate state-value approximation, since it can be directly computed
from a soft-Q function and reference policy, 
using \cref{soft-v-and-pi}.
However, including a separate function approximator for the state-value
can help stabilize the training \citep{haarnoja2018soft}. 
The soft Q-function parameters $\psi$ is then trained to minimize the soft Bellman error using the state-value network,
\begin{equation}
\label{eq:trainQ}
L(\psi) = \E_{(s,a,s') \sim \gD} \left[ \frac{1}{2} \left( \parQ(s,a) - \left[r(s,a) + \gamma \parV(s') \right] \right)^2 \right]
\end{equation}

The policy parameters are updated by performing the Project Step in
\cref{eq:repmd} with stochastic gradient descent,
%
\begin{equation}
\label{eq:trainpi}
L(\theta) = \E_{s \sim \gD} \left[ \KL \left( \exp\left\{ \frac{\parQ(s,\cdot) + \tau \log \refPi(\cdot|s) - \parV(s)}{\tau+\tau'} \right\} \middle\Vert \parPi(\cdot |s) \right) \right]
\end{equation}
%
where we approximate $\bar{\pi}_{\tau,\tau^{\prime}}^*$ by the soft-Q and state-value function approximations. 

Finally,
we also use a target state-value network \citep{lillicrap2015continuous}
and the trick of maintaining two soft-Q functions
\citep{haarnoja2018soft,fujimoto2018addressing}.
Implementation details for PMAC are given in \cref{sec:implementationPMAC}. 


