


%\subsection{Cooperate with Value Function Approximation}
\section{Cooperate with Value Function Approximation}
\label{subsec:repmd_value}

\newcommand{\parV}{V_{\phi}}
\newcommand{\parTargetV}{V_{\bar{\phi}}}
\newcommand{\parQ}{Q_{\psi}}
\newcommand{\parPi}{\pi_{\theta}}
\newcommand{\parQone}{Q_{\psi_1}}
\newcommand{\parQtwo}{Q_{\psi_2}}


The last part of this section is a nature extension of REPMD with value function approximation, named as Policy Mirror Actor-Critic (PMAC) . 



It has been shown that the data efficiency of policy-based methods can be improved by a critic in general. 
In this section, we present an actor-critic version of REPMD, named as Policy Mirror Actor Critic (PMAC). 
%We consider an infinite trajectory $\rho$ while keeping the discounted factor $\gamma=1$ for simplicity. All the results have no difficulty to extend to a  general value of $\gamma$. 
Given a reference policy $\refPi$ and an initial state $s$, recall that the objective in the lift step is 
\[
\gO_{\text{REPMD}}(\pi, s) =   \ep\limits_{\rho \sim \pi}{  r(\rho)  - \tau \KL(\pi \| \refPi) + \tau^{\prime} \cH(\pi)}, 
\]
where $\rho=  (s_1 = s, a_1, s_2, a_2, \ldots)$.
 To cooperate with value function approximation, we will need to derive the temporal consistency for this objective, as follows. 
 \[
 \gO_{\text{REPMD}}(\pi, s) = \E_{a\sim \pi(\cdot \vert s)} \left[ r(s,a) + \gO_{\text{REPMD}}(\pi, s')  + \tau \log \refPi(a|s) - \left(\tau+ \tau' \right) \log \pi(a|s) \right]. 
 \]
%We first alter our definitions of policy and value functions to include the KL and entropy regularization. Given a reference policy $\refPi(a|s)$, we consider the entropy regularized maximum reward objective, \todoc[]{add the trust-pcl reference here or just discuss it in related work?}
%\begin{equation*}
%\gO_{\text{RELENT}}(\pi, s) = \E_{a\sim \pi} \left[ r(s,a) + \gO_{\text{RELENT}}(\pi, s')  + \tau \log \refPi(a|s) - \left(\tau+ \tau' \right) \log \pi(a|s) \right] \\
%\label{relent-obj}
%\end{equation*}
Let $\bar{\pi}_{\tau,\tau^{\prime}}^* (\cdot|s) = \argmax_{\pi} \gO_{\text{REPMD}}(\pi, s) $ denote the optimal policy on state $s$. 
%By \cref{eq:pitautauprime}, 
%\[
%\bar{\pi}_{\tau,\tau^{\prime}}^* (a|s) = \frac{1}{Z} \exp \left\{ \frac{\left[ r(s,a) + \gO_{\text{REPMD}}(\bar{\pi}_{\tau,\tau^{\prime}}^* (\cdot|s'), s') \right] + \tau \log \bar{\pi}(a|s) }{\tau + \tau'} \right\}.
%\]
Further denote the soft optimal state value function $\gO_{\text{REPMD}}(\bar{\pi}_{\tau,\tau^{\prime}}^* (\cdot|s), s)$ by $\bar{V}_{\tau,\tau^{\prime}}^*(s)$, and let  $\bar{Q}_{\tau,\tau^{\prime}}^*(s,a) = r(s,a) + \gamma \bar{V}_{\tau,\tau^{\prime}}^*(s')$ be the soft-Q function.
 It can be verified that 
\begin{equation}
\begin{split}
& \bar{V}_{\tau,\tau^{\prime}}^*(s) = (\tau + \tau') \log \sum_a \exp \left\{ \frac{\bar{Q}_{\tau,\tau^{\prime}}^*(s,a) + \tau \log \bar{\pi}(a|s)} {\tau + \tau'} \right\}; \\
& \bar{\pi}_{\tau,\tau^{\prime}}^* (a|s) = \exp \left\{ \frac{\bar{Q}_{\tau,\tau^{\prime}}^*(s,a) + \tau \log \bar{\pi}(a|s) - \bar{V}_{\tau,\tau^{\prime}}^*(s)}{\tau + \tau'} \right\}.
\end{split}
\label{soft-v-and-pi}
\end{equation}
 
% The soft optimal state value can be defined by $\bar{V}_{\tau,\tau^{\prime}}^*(s) = \gO_{\text{RELENT}}(\bar{\pi}_{\tau,\tau^{\prime}}^*, s)$. According to \cref{eq:pitautauprime}, both $\bar{\pi}_{\tau,\tau^{\prime}}^*(\cdot | s)$ and $\bar{V}_{\tau,\tau^{\prime}}^*(s)$ have closed form solution that,

%\subsubsection{Learning}
We propose to train a soft state value function $\parV$ parameterized by $\phi$, a soft Q-function $\parQ$ parameterized by $\psi$, and a policy $\parPi$ parameterized by $\theta$, based on \cref{eq:repmd}. We now derive the update rules for these parameters. 

The soft state value function approximates the soft optimal state value $\bar{V}_{\tau,\tau^{\prime}}^*$. Note that we can re-express $\bar{V}_{\tau,\tau^{\prime}}^*$ by 
\begin{align*}
\bar{V}_{\tau,\tau^{\prime}}^*(s) 
%= & (\tau + \tau') \log \sum_a \refPi(a|s) \exp \left\{ \frac{\bar{Q}_{\tau,\tau^{\prime}}^*(s,a) - \tau' \log \bar{\pi}(a|s)} {\tau + \tau'} \right\} \\ 
= & (\tau + \tau') \log \E _ {a\sim \refPi} \left[ \exp \left\{ \frac{\bar{Q}_{\tau,\tau^{\prime}}^*(s,a) - \tau' \log \bar{\pi}(a|s)} {\tau + \tau'} \right\} \right].
\end{align*}
This suggests a Monte-Carlo estimation of $\bar{V}_{\tau,\tau^{\prime}}^*(s)$: by sampling one single action $a$ according to the reference policy $\refPi$, we have $\bar{V}_{\tau,\tau^{\prime}}^*(s) \approx  \bar{Q}_{\tau,\tau^{\prime}}^*(s,a) - \tau' \log \bar{\pi}(a|s) $. Then the soft state value function is trained to minimize the mean squared error,
\begin{equation}
\label{eq:trainV}
L (\phi) = \E_{s\sim \gD} \left[ \frac{1}{2} \left( \parV(s) -  \left[ \parQ(s,a ) - \tau' \log \bar{\pi}(a|s) \right] \right)^2 \right]
\end{equation}
where $\gD$ is a reply buffer. 

One may note that there is no need in principle to include a separate state value function approximation, since it can be computed directly given a soft-Q function and reference policy according to (\ref{soft-v-and-pi}). However, including a separate function approximation for the state value can help stabilize the training \citep{haarnoja2018soft}. 
%Furthermore, to increase the stability of the training, we include a target state value network $\parTargetV$, where $\bar{\phi}$ is an exponentially moving average of the value network weights $\phi$. 
The soft Q-function parameters $\psi$ is then trained to minimize the soft Bellman error using the state value network,
\begin{equation}
\label{eq:trainQ}
L(\psi) = \E_{(s,a,s') \sim \gD} \left[ \frac{1}{2} \left( \parQ(s,a) - \left[r(s,a) + \gamma \parV(s') \right] \right)^2 \right]
\end{equation}

Finally, the policy parameters is updated by doing the project step in (\ref{eq:repmd}) with stochastic gradient descent,
\begin{equation}
\label{eq:trainpi}
L(\theta) = \E_{s \sim \gD} \left[ \KL \left( \exp\left\{ \frac{\parQ(s,\cdot) + \tau \log \refPi(\cdot|s) - \parV(s)}{\tau+\tau'} \right\} \middle\Vert \parPi(\cdot |s) \right) \right]
\end{equation}
%\begin{align*}
%\gO_{\text{RMAC}} (\theta) = \E_{s \sim \gD} \left[ \KL \left( \frac{\refPi(a|s) \exp\left\{ \frac{\parQ(s,a) - \tau' \log \refPi(a|s)}{\tau+\tau'} \right\}}{\exp \left\{ \frac{\parV(s)}{\tau+\tau'} \right\}}  \middle\Vert \parPi \right) \right]
%\end{align*}
where we approximate $\bar{\pi}_{\tau,\tau^{\prime}}^*$ by the soft-Q and state value function approximations. 
%The next theorem shows that the gradient of \cref{eq:trainpi} can be computed by importance sampling using the reference policy, 
%\begin{thm}
%\label{thm:rmacgradientestimate}
%The gradient of \cref{eq:trainpi} is,
%\begin{equation}
%\label{eq:mac_gradient_estimator}
%	\nabla_\theta L(\theta) = \nabla_\theta \E_{s \sim \gD} \left[ \E_{a\sim \refPi}\left[  \exp\left\{ \frac{\parQ(s,a) - \tau' \log \refPi(a|s) - \parV(s)}{\tau+\tau'} \right\} \log \parPi(a |s) \right]   \right].
%\end{equation}
%\end{thm}
%\begin{proof}
%Let $\pi(a \vert s) =  \exp\left\{ \frac{\parQ(s,a) + \tau \log \refPi(a|s) - \parV(s)}{\tau+\tau'} \right\}$, then we have,
%\begin{align*}
%\nabla_\theta L(\theta) = & \nabla_\theta  \E_{s \sim \gD} \left[\sum_a  \pi(a|s) \log \pi(a|s) - \pi(a|s) \log \pi_\theta(a|s) \right]\\
%= & \nabla_\theta \E_{s \sim \gD} \left[ -\sum_a \exp\left\{ \frac{\parQ(s,a) + \tau \log \refPi(a|s) - \parV(s)}{\tau+\tau'} \right\} \log \parPi(a |s) \right] \\ 
%= & \nabla_\theta \E_{s \sim \gD} \left[ -\sum_a \refPi(a|s) \exp\left\{ \frac{\parQ(s,a) - \tau' \log \refPi(a|s) - \parV(s)}{\tau+\tau'} \right\} \log \parPi(a |s) \right] \\ 
%= & \nabla_\theta \E_{s \sim \gD} \left[ \E_{a\sim \refPi}\left[ - \exp\left\{ \frac{\parQ(s,a) - \tau' \log \refPi(a|s) - \parV(s)}{\tau+\tau'} \right\} \log \parPi(a |s) \right]   \right] 
%\end{align*}
%\end{proof}

%Our approach also use two soft-Q functions in order to mitigate the overestimation problem caused by value function approximation \citep{haarnoja2018soft,fujimoto2018addressing}. Specifically, we apply two soft-Q function approximations, $\parQone(s,a)$ and $\parQtwo(s,a)$, and train them independently.  The minimum of the two Q-functions will be used whenever the soft-Q value is needed. 
%The complete algorithm is described in Algorithm \ref{}.

Lastly, we also use a target state value network \citep{lillicrap2015continuous} and the trick of two soft-Q functions \citep{haarnoja2018soft,fujimoto2018addressing}. Implementation details of PMAC is described in \cref{sec:implementationPMAC}. 

%\begin{align*}
%\gO_{\text{RMAC}} (\theta) = & \E_{s \sim \gD} \left[ \KL \left( \bar{\pi}_{\tau,\tau^{\prime}}^*(\cdot \vert s)  \Vert \pi_\theta(\cdot \vert s) \right) \right] \\
%= & \E_{s \sim \gD} \left[ \E_{a\sim \refPi} \left[ \frac{\bar{\pi}_{\tau,\tau^{\prime}}^* (a|s)}{\refPi(a|s)} \left( \log  \bar{\pi}_{\tau,\tau^{\prime}}^* (a|s) - \log \pi_\theta (a|s) \right) \right] \right] \\
%= & \E_{s \sim \gD} \left[ \E_{a\sim \refPi} \left[ \frac{\refPi(a|s) \exp\left\{ \frac{\parQ(s,a) - \tau' \log \refPi(a|s)}{\tau+\tau'} \right\}}{\exp \left\{ \frac{\parV(s)}{\tau+\tau'} \right\}} \left( \log  \bar{\pi}_{\tau,\tau^{\prime}}^* (a|s) - \log \pi_\theta (a|s) \right) \right] \right]
%\end{align*}

