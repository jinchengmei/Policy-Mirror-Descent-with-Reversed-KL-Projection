\section{Proofs for \cref{sec:reversed_emtropy_policy_mirror_descent}}
This section is devoted to the omitted proofs in \cref{sec:reversed_emtropy_policy_mirror_descent}.
\subsection{Proof of \cref{prop:mirrordescent_projection}}
\begin{proof}
	Note that $-\tau \KL(\pi_\theta \| \bar{\pi}_\tau^*) = - \tau \sum_{\rho}{ \pi_\theta(\rho) \log \pi_\theta(\rho) } + \tau \sum_{\rho}{ \pi_\theta(\rho) (\log \refPi(\rho) + r(\rho) / \tau ) }  - Z_{\refPi} = \ep_{\rho \sim \pi_\theta}{  r(\rho)  - \tau \KL(\pi_\theta \| \refPi) } - Z_{\refPi}$. Note the fact that $Z_{\refPi} \triangleq \tau \log{ \sum_{\rho}{\refPi(\rho) \exp\left\{ r(\rho) / \tau \right\} } }$ is indenpendent of $\pi_\theta$ given the reference policy $\refPi$.
\end{proof}

\subsection{Proof of \cref{prop:monoto_policymirrordescent}}
\label{appsec:monoto_policymirrordescent}
\begin{proof}
	{\bf (Monotonic Improvement Guarantee)} By the definition of $\pi_{\theta_{t+1}}$, note that $\KL(\pi_{\theta_{t+1}} \| \bar{\pi}_\tau^*)  = \min_{\pi_\theta \in \Pi}{ \KL(\pi_\theta \| \bar{\pi}_\tau^*)} \leq \KL(\pi_{\theta_{t}} \| \bar{\pi}_\tau^*)$. By expanding the KL divergence and rearranging terms, we have $ \tau \KL(\pi_{\theta_{t+1}} \| \pi_{\theta_{t}}) - \sum_{\rho}{ \pi_{\theta_{t+1}}(\rho) r(\rho) } \leq - \sum_{\rho}{ \pi_{\theta_{t}}(\rho) r(\rho) }$, which gives $\ep_{\rho \sim \pi_{\theta_{t+1}}}{  r(\rho)} - \ep_{\rho \sim \pi_{\theta_{t}}}{  r(\rho)} \geq \tau \KL(\pi_{\theta_{t+1}} \| \pi_{\theta_{t}}) \geq 0$.
	
	{\bf (Global optimum inclusion)}
\end{proof}

\subsection{Proof of \cref{thm:monotonically_increasing_sr_property}}
\begin{proof}
	Using $\KL(\bar{\pi}_{\tau,\tau^{\prime}}^* \| \pi_{\theta_{t+1}}) = \min_{\pi_\theta \in \Pi}{ \KL(\bar{\pi}_{\tau,\tau^{\prime}}^* \| \pi_\theta)} \le \KL(\bar{\pi}_{\tau,\tau^{\prime}}^* \| \pithetat)$ and Jensen's inequality,
	\begin{equation*}
	\begin{split}
	&\SR(\pi_{\theta_{t+1}}) - \SR(\pithetat) = (\tau + \tau^{\prime}) \log{ \sum\limits_{\rho}{ \frac{  \exp\left\{ \frac{r(\rho) + \tau \log{\pi_{\theta_{t+1}}(\rho)} }{\tau + \tau^{\prime}} \right\}  }{ \sum\limits_{\rho}{  \exp\left\{ \frac{r(\rho) + \tau \log{\pithetat(\rho)} }{\tau + \tau^{\prime}} \right\} } }  } } \\
	=& (\tau + \tau^{\prime}) \log{ \sum\limits_{\rho}{ \frac{  \exp\left\{ \frac{r(\rho) + \tau \log{\pithetat(\rho)} }{\tau + \tau^{\prime}} \right\}  }{ \sum\limits_{\rho}{  \exp\left\{ \frac{r(\rho) + \tau \log{\pithetat(\rho)} }{\tau + \tau^{\prime}} \right\} } }  } \cdot \exp\left\{ \frac{\tau \log{\pi_{\theta_{t+1}}(\rho)} - \tau \log{\pithetat(\rho)} }{\tau + \tau^{\prime}} \right\} } \\
	=& (\tau + \tau^{\prime}) \log{ \sum\limits_{\rho}{ \bar{\pi}_{\tau,\tau^{\prime}}^*(\rho) } \cdot \exp\left\{ \frac{\tau \log{\pi_{\theta_{t+1}}(\rho)} - \tau \log{\pithetat(\rho)} }{\tau + \tau^{\prime}} \right\} } \\
	\ge& (\tau + \tau^{\prime}) \sum\limits_{\rho}{ \bar{\pi}_{\tau,\tau^{\prime}}^*(\rho) \log{ \exp\left\{ \frac{\tau \log{\pi_{\theta_{t+1}}(\rho)} - \tau \log{\pithetat(\rho)} }{\tau + \tau^{\prime}} \right\} } } \\
	=& \tau \sum\limits_{\rho}{ \bar{\pi}_{\tau,\tau^{\prime}}^*(\rho) \log{ \frac{\pi_{\theta_{t+1}}(\rho)}{\pithetat(\rho)} } } = \tau \left[ \KL(\bar{\pi}_{\tau,\tau^{\prime}}^* \| \pithetat) - \KL(\bar{\pi}_{\tau,\tau^{\prime}}^* \| \pi_{\theta_{t+1}})\right] \ge 0. \qedhere
	\end{split}
	\end{equation*}
\end{proof}

\subsection{Proof of \cref{prop:solvableprojection}}
\begin{proof}
	Note that $\pi_{\theta} = \frac{\exp\left\{\Phi^\top\theta\right\}}{{\mathbf{1}^\top \exp\left\{\Phi^\top\theta\right\} }}$, where $\Phi$ is the feature matrix and $\theta$ is the policy parameter.
	Simply compute the Hessian matrix of the objective,
	 \[\frac{d^2 \KL(\refPi \| \pi_{\theta})}{d \theta^2} = \Phi (\Delta(\pi_\theta) - \pi_\theta \pi_\theta^\top) \Phi^\top \succeq 0.
	 \]
	 Thus $\KL(q \| \pi_{\theta})$ is convex in $\theta$.
\end{proof}

\subsection{Proof of \cref{prop:sr}}
\begin{proof}
	To prove (i), note that as $\tau \to 0$, $\SR(\pi_\theta) \to \tau^{\prime} \log{ \sum_{\rho}{ \exp\left\{ \frac{r(\rho) }{ \tau^{\prime} } \right\} }}$, the standard softmax value. Taking limit on $\tau'$ gives the hardmax value $\max_{\rho}{r(\rho)}$ as $\tau^{\prime} \to 0$.
	
	To prove (ii), we have 
	\begin{align*}
	&\lim\limits_{\tau \to \infty}{ (\tau + \tau^{\prime})\log{ \sum_{\rho}{ \exp\left\{ \frac{r(\rho) + \tau \log{\pi_\theta(\rho)} }{\tau + \tau^{\prime}} \right\} }} } = {\scalebox{.95} {$\lim\limits_{\tau \to \infty}{ \frac{ \sum_{\rho}{ \pi_\theta(\rho) \exp\{ \frac{r(\rho) - \tau^{\prime} \log{\pi_\theta(\rho)} }{\tau + \tau^{\prime}} \} \left( r(\rho) - \tau^{\prime}\log{\pi_\theta(\rho)} \right) } }{  \sum_{\rho}{ \pi_\theta(\rho) \exp\{ \frac{r(\rho) - \tau^{\prime} \log{\pi_\theta(\rho)} }{\tau + \tau^{\prime}} \} } } }$ } }\\
	&= \sum_{\rho}{ \pi_\theta(\rho) \left[ r(\rho) - \tau^{\prime}\log{\pi_\theta(\rho)} \right] } = \ep_{\rho \sim \pi_\theta}{r(\rho)} + \tau^{\prime} \mathcal{H}(\pi_\theta)
	\end{align*}
	As $\tau^{\prime} \to 0$, $\SR(\pi_\theta) \to \ep_{\rho \sim \pi_\theta}{r(\rho)}$.
\end{proof}

\subsection{Proof for Section ***}
\begin{lem}
	\label{lem:opt_pi_ref}
	The lift step of \cref{eq:repmd} has the following closed form expression:
	\begin{equation*}
	\bar{\pi}_{\tau,\tau^{\prime}}^*(\rho) \triangleq \frac{\refPi(\rho) \exp\left\{ \frac{r(\rho)-\tau^{\prime} \log \refPi(\rho) }{ \tau+\tau^{\prime}} \right\}}{ \sum_{\rho^{\prime}}{\refPi(\rho^{\prime}) \exp\left\{ \frac{r(\rho^{\prime})-\tau^{\prime} \log \refPi(\rho^{\prime})}{ \tau+\tau^{\prime}} \right\} } }.
	\end{equation*}
\end{lem}
\begin{proof}
	Rewrite the objective function defined in \cref{eq:repmd},
	\begin{equation}
	\ep\limits_{\rho \sim \pi} r(\rho)  - \tau \KL(\pi \| \refPi) + \tau^{\prime} \cH(\pi) = \ep\limits_{\rho \sim \pi} [r(\rho) + \tau \log \refPi(\rho)] + (\tau+\tau^{\prime}) \cH(\pi),
	\end{equation}
	which is an entropy regularized reshaped reward objective. The optimal policy of this objective can be obtained by directly applying Lemma 4 of \citet{nachum2017bridging}, i.e.
	\begin{equation}
	\label{pi_bar_star_tau_tauprime_prop_form}
	\bar{\pi}_{\tau,\tau^{\prime}}^*(\rho)\propto \exp\left\{ \frac{r(\rho)+\tau \log \refPi(\rho)}{\tau+\tau^{\prime}} \right\} = \refPi(\rho) \exp\left\{ \frac{r(\rho)-\tau^{\prime} \log \refPi(\rho)}{\tau+\tau^{\prime}} \right\}. \qedhere
	\end{equation}
\end{proof}

\subsection{Proof of \cref{thm:repmdgradientestimate}}
\begin{proof}
	Note that
	\begin{equation*}
	\label{eq:importance_sampling_kl}
	\KL(\bar{\pi}_{\tau,\tau^{\prime}}^* \| \pi_\theta) = \ep_{\rho \sim \bar{\pi}_{\tau,\tau^{\prime}}^* } \left[ \log \bar{\pi}_{\tau,\tau^{\prime}}^*(\rho) - \log \pi_\theta(\rho) \right] = \ep_{\rho\sim \refPi} \left[  \frac{\bar{\pi}_{\tau,\tau^{\prime}}^*(\rho)}{\refPi(\rho)} \left(\log \bar{\pi}_{\tau,\tau^{\prime}}^*(\rho) - \log \pi_\theta(\rho) \right) \right].
	\end{equation*}	
	%	We can draw $K$ \textit{i.i.d.} samples $\{\rho_1, \dots, \rho_K\}$ from the \emph{reference policy} $\refPi$, and then approximate the gradient of $\KL(\bar{\pi}_{\tau,\tau^{\prime}}^* \| \pi_\theta)$ by averaging these $K$ samples according to \cref{eq:importance_sampling_kl}. 
	Therefore, taking gradient on both sides,
	\begin{equation}
	\begin{split}
	&\nabla_{\theta} \KL(\bar{\pi}_{\tau,\tau^{\prime}}^* \| \pi_\theta) \approx -\frac{1}{K}\sum_{k=1}^K \frac{\bar{\pi}_{\tau,\tau^{\prime}}^* (\rho_k)}{\refPi(\rho_k)} \nabla_{\theta} \log \pi_\theta(\rho_k) \\ 
	\approx & -\frac{1}{K}\sum_{k=1}^K \frac{\exp \{\omega_k\}} {\frac{1}{K} \sum_{j=1}^K \exp \{\omega_j\}} \nabla_{\theta} \log \pi_\theta(\rho_k) =  -\sum\limits_{k=1}^K{ \frac{ \exp\left\{ \omega_k \right\} }{ \sum_{j=1}^K{ \exp\left\{ \omega_j \right\}}} \nabla_{\theta} \log{\pi_\theta(\rho_k)} } \qedhere.
	\end{split}
	\end{equation}
	\todor[]{More details}
\end{proof}

\section{Stochastic Transition Setting}
\label{sec:stochasticsetting}
In \cref{sec:notations_and_settings}, we assume that the state transition function is deterministic for simplicity. For completeness, we consider the general stochastic transition setting here.

\subsection{Notations and Settings}

Recall in \cref{sec:notations_and_settings}, the policy probability of trajectory $\rho=(s_1, a_1, \dots, a_{T-1}, s_T)$ is denoted as $\pi(\rho) = \prod_{t=1}^{T-1} \pi(a_t| s_t)$. We define transition probability of $\rho$ as $f(\rho) \triangleq \prod_{t=1}^{T-1}{ f(s_{s+1} | s_t, a_t)}$. The total probability of $\rho$ under policy $\pi$ and transition $f$ is then $p_{\pi, f}(\rho) \triangleq \pi(\rho) f(\rho) = \prod_{t=1}^{T-1}{ \pi(a_t | s_t) f(s_{s+1} | s_t, a_t)}$. We use $\Delta_{f} \triangleq \{ \pi | \sum_{\rho}{ p_{\pi, f}(\rho) } = \sum_{\rho}{\pi(\rho) f(\rho)} = 1, \pi(\rho) \ge 0, f(\rho) > 0, \forall \rho \}$ to refer to the probabilistic simplex over all possible trajectories. It is obvious that $p_{\pi, f}(\rho) = \pi(\rho)$ and $\Delta_f = \Delta$ under deterministic transition setting, i.e., $f(\rho) = 1, \forall \rho$.

\subsection{REPMD Optimization Problem}

The proposed REPMD algorithm solves \cref{eq:repmd} in the deterministic transition setting. In the stochastic setting, the corresponding problem is,
\begin{equation}
\label{eq:repmd_stochastic}
\begin{split}
	&\argmin\limits_{\pi_\theta \in \Pi}{\KL(p_{\bar{\pi}_{\tau,\tau^{\prime}}^*, f}  \| p_{\pi_\theta, f} ) }, \\
	\text{where}\ \ \bar{\pi}_{\tau,\tau^{\prime}}^* & =  \argmax\limits_{\pi \in \Delta_f}{ \ep\limits_{\rho \sim p_{\pi, f}}{ \left[ r(\rho) - \tau^{\prime} \log{\pi(\rho)} \right] } - \tau \KL(p_{\pi, f} \| p_{\pi_{\theta_t}, f} ) },
\end{split}
\end{equation}
which also recovers \cref{eq:repmd} as a special case when $f(\rho) = 1, \forall \rho$.

Like \cref{eq:repmd}, $\bar{\pi}_{\tau,\tau^{\prime}}^*$ in \cref{eq:repmd_stochastic} also has a closed form expression,
\begin{lem}
\label{lem:opt_pi_ref_stochastic}
The unconstrained optimal policy of \cref{eq:repmd_stochastic} has the following closed form expression:
\begin{equation*}
	\bar{\pi}_{\tau,\tau^{\prime}}^*(\rho) \triangleq \frac{\refPi(\rho) \exp\left\{ \frac{r(\rho)-\tau^{\prime} \log \refPi(\rho) }{ \tau+\tau^{\prime}} \right\}}{ \sum_{\rho^{\prime}}{\refPi(\rho^{\prime}) f(\rho^{\prime}) \exp\left\{ \frac{r(\rho^{\prime})-\tau^{\prime} \log \refPi(\rho^{\prime})}{ \tau+\tau^{\prime}} \right\} } }.
\end{equation*}
\end{lem}
\begin{proof}
Rewrite the maximization problem in \cref{eq:repmd_stochastic} as (take $\pi_{\theta_t}$ as the reference policy $\refPi$),
\begin{equation*}
\begin{split}
	\textmax\limits_{\pi}&{ \sum\limits_{\rho}{ \pi(\rho) f(\rho) \left[ r(\rho)  - \left( \tau + \tau^{\prime} \right) \log{\pi(\rho)} + \tau \log{\bar{\pi}(\rho)} \right]} } \\
	\st &\sum\limits_{\rho}{ \pi(\rho) f(\rho)} = 1.
\end{split}
\end{equation*}
The KKT condition of the above problem is,
\begin{equation*}
\begin{split}
	f(\rho) \left[ r(\rho) - \left( \tau + \tau^{\prime} \right) \log{\pi(\rho)} + \tau \log{\bar{\pi}(\rho)} +  \lambda - \left( \tau + \tau^{\prime} \right) \right] &= 0, \ \forall \rho \\
	\sum\limits_{\rho}{ \pi(\rho) f(\rho)} &= 1.
\end{split}
\end{equation*}
Using $f(\rho) > 0, \forall \rho$ and solving the KKT condition, we obtain the expression of $\bar{\pi}_{\tau,\tau^{\prime}}^*$.
\end{proof}

Lemma \ref{lem:opt_pi_ref_stochastic} recovers Lemma \ref{lem:opt_pi_ref} as a special case when $f(\rho) = 1, \forall \rho$.

\subsection{Theoretical Analysis}

In stochastic transition setting, we define the follow softmax approximated expected reward of $\pi_\theta$
\begin{equation*}
	\SR_f(\pi_\theta) \triangleq (\tau + \tau^{\prime})\log{ \sum_{\rho}{ f(\rho) \exp\left\{ \frac{r(\rho) + \tau \log{\pi_\theta(\rho)} }{\tau + \tau^{\prime}} \right\} }},
\end{equation*}
which recovers $\SR(\pi_\theta)$ when $f(\rho) = 1, \forall \rho$. The monotonic improvement property is for $\SR_f(\pi_\theta)$.

\begin{thm}
\label{thm:monotonically_increasing_sr_property_stochastic}
Assume that $\pi_{\theta_{t}}$ is the update sequence of the REPMD algorithm in \cref{eq:repmd_stochastic}, then
\begin{equation*}
	\SR_f(\pi_{\theta_{t+1}}) - \SR_f(\pithetat)\ge 0.
\end{equation*}
\end{thm}
\begin{proof}
	Using $\KL(p_{\bar{\pi}_{\tau,\tau^{\prime}}^*, f} \| p_{\pi_{\theta_{t+1}}, f}) = \min_{\pi_\theta \in \Pi}{ \KL(p_{\bar{\pi}_{\tau,\tau^{\prime}}^*, f}  \| p_{\pi_\theta, f} ) } \le \KL(p_{\bar{\pi}_{\tau,\tau^{\prime}}^*, f} \| p_{\pithetat, f})$ and Jensen's inequality,
	\begin{equation*}
	\begin{split}
	&\SR_f(\pi_{\theta_{t+1}}) - \SR_f(\pithetat) = (\tau + \tau^{\prime}) \log{ \sum\limits_{\rho}{ \frac{ f(\rho) \exp\left\{ \frac{r(\rho) + \tau \log{\pi_{\theta_{t+1}}(\rho)} }{\tau + \tau^{\prime}} \right\}  }{ \sum\limits_{\rho}{ f(\rho) \exp\left\{ \frac{r(\rho) + \tau \log{\pithetat(\rho)} }{\tau + \tau^{\prime}} \right\} } }  } } \\
	=& (\tau + \tau^{\prime}) \log{ \sum\limits_{\rho}{ \frac{ f(\rho) \exp\left\{ \frac{r(\rho) + \tau \log{\pithetat(\rho)} }{\tau + \tau^{\prime}} \right\}  }{ \sum\limits_{\rho}{ f(\rho) \exp\left\{ \frac{r(\rho) + \tau \log{\pithetat(\rho)} }{\tau + \tau^{\prime}} \right\} } }  } \cdot \exp\left\{ \frac{\tau \log{\pi_{\theta_{t+1}}(\rho)} - \tau \log{\pithetat(\rho)} }{\tau + \tau^{\prime}} \right\} } \\
	=& (\tau + \tau^{\prime}) \log{ \sum\limits_{\rho}{ \bar{\pi}_{\tau,\tau^{\prime}}^*(\rho) f(\rho) } \cdot \exp\left\{ \frac{\tau \log{\pi_{\theta_{t+1}}(\rho)} - \tau \log{\pithetat(\rho)} }{\tau + \tau^{\prime}} \right\} } \\
	\ge& (\tau + \tau^{\prime}) \sum\limits_{\rho}{ \bar{\pi}_{\tau,\tau^{\prime}}^*(\rho) f(\rho) \cdot \log{ \exp\left\{ \frac{\tau \log{\pi_{\theta_{t+1}}(\rho)} - \tau \log{\pithetat(\rho)} }{\tau + \tau^{\prime}} \right\} } } \\
	=& \tau \sum\limits_{\rho}{ \bar{\pi}_{\tau,\tau^{\prime}}^*(\rho) f(\rho) \cdot \log{ \frac{\pi_{\theta_{t+1}}(\rho)}{\pithetat(\rho)} } } \\
	=& \tau \sum\limits_{\rho}{ \bar{\pi}_{\tau,\tau^{\prime}}^*(\rho) f(\rho) \cdot \log{ \frac{ \pi_{\theta_{t+1}}(\rho) f(\rho) }{ \pithetat(\rho) f(\rho)} } } \\
	=& \tau \left[ \KL(p_{\bar{\pi}_{\tau,\tau^{\prime}}^*, f} \| p_{\pithetat, f}) - \KL(p_{\bar{\pi}_{\tau,\tau^{\prime}}^*, f} \| p_{\pi_{\theta_{t+1}}, f}) \right] \ge 0. \qedhere
	\end{split}
	\end{equation*}
\end{proof}

$\SR_f(\pi_\theta)$ also recovers corresponding performance measures in the stochastic transition setting.
\begin{prop}
\label{prop:sr_stochastic}
$\SR_f(\pi_\theta)$ satisfies the following properties:
\begin{enumerate}[label=(\roman*)]
	\item  $\SR_f(\pi_\theta) \to \max_{\rho}{r(\rho)}$, as $\tau \to 0, \tau^{\prime} \to 0$.
	\item $\SR_f(\pi_\theta) \to \ep\limits_{\rho \sim p_{\pi_\theta, f}}{r(\rho)}$, as $\tau \to \infty, \tau^{\prime} \to 0$. 
\end{enumerate}	
\end{prop}
\begin{proof}
To prove (i), note that as $\tau \to 0$, $\SR_f(\pi_\theta) \to \tau^{\prime} \log{ \sum_{\rho}{ f(\rho) \exp\left\{ \frac{r(\rho) }{ \tau^{\prime} } \right\} }}$. Taking limit on $\tau'$ gives the hardmax value $\max_{\rho}{r(\rho)}$ as $\tau^{\prime} \to 0$.
	
To prove (ii), we have 
\begin{align*}
	&\lim\limits_{\tau \to \infty}{ (\tau + \tau^{\prime})\log{ \sum_{\rho}{ f(\rho) \exp\left\{ \frac{r(\rho) + \tau \log{\pi_\theta(\rho)} }{\tau + \tau^{\prime}} \right\} }} } \\
	=& \lim\limits_{\tau \to \infty}{ \frac{ \sum_{\rho}{ \pi_\theta(\rho) f(\rho) \exp\{ \frac{r(\rho) - \tau^{\prime} \log{\pi_\theta(\rho)} }{\tau + \tau^{\prime}} \} \left( r(\rho) - \tau^{\prime}\log{\pi_\theta(\rho)} \right) } }{  \sum_{\rho}{ \pi_\theta(\rho) f(\rho) \exp\{ \frac{r(\rho) - \tau^{\prime} \log{\pi_\theta(\rho)} }{\tau + \tau^{\prime}} \} } } } \\
	=& \sum_{\rho}{ \pi_\theta(\rho) f(\rho) \left[ r(\rho) - \tau^{\prime}\log{\pi_\theta(\rho)} \right] } = \ep\limits_{\rho \sim p_{\pi_\theta, f}}{r(\rho)} -  \tau^{\prime} \cdot \ep\limits_{\rho \sim p_{\pi_\theta, f}}{  \log{\pi_\theta(\rho)} }
\end{align*}
As $\tau^{\prime} \to 0$, $\SR_f(\pi_\theta) \to \ep_{\rho \sim p_{\pi_\theta, f}}{r(\rho)}$.
\end{proof}

\subsection{Learning}

The REPMD learning process is intact under the stochastic transition setting. Similar with \cref{eq:importance_sampling_kl}, we can estimate the KL divergence in the projection step of \cref{eq:repmd_stochastic} by drawing $K$ \textit{i.i.d.} samples $\{\rho_1, \dots, \rho_K\}$ from $p_{\refPi, f}$, i.e., the mixture of $\refPi$ and $f$, which is exactly the process of sampling from $\refPi$ and interacting with the environment,
\begin{equation}
\label{eq:importance_sampling_kl_stochastic}
\begin{split}
	\KL(p_{\bar{\pi}_{\tau,\tau^{\prime}}^*, f}  \| p_{\pi_\theta, f} ) &= \ep_{\rho \sim p_{\bar{\pi}_{\tau,\tau^{\prime}}^*, f} } \left[ \log \bar{\pi}_{\tau,\tau^{\prime}}^*(\rho) - \log \pi_\theta(\rho) \right] \\
	&= \ep_{\rho\sim p_{\refPi, f}} \frac{\bar{\pi}_{\tau,\tau^{\prime}}^*(\rho)}{\refPi(\rho)} \left[ \log \bar{\pi}_{\tau,\tau^{\prime}}^*(\rho) - \log \pi_\theta(\rho) \right].
\end{split}
\end{equation}

We can then approximate the gradient of $\KL(p_{\bar{\pi}_{\tau,\tau^{\prime}}^*, f}  \| p_{\pi_\theta, f} )$ by averaging these $K$ samples according to \cref{eq:importance_sampling_kl_stochastic}. 

\begin{thm}
\label{thm:repmdgradientestimate_stochastic}
Let $\omega_k = \frac{r(\rho_k) - \tau^{\prime} \log{\refPi(\rho_k)} }{\tau + \tau^{\prime}}$. Given $K$ \emph{i.i.d.} samples $\{\rho_1, \dots, \rho_K\}$ from the \emph{reference policy} $\refPi$, we have the following unbiased gradient estimator,
\begin{equation}
	\nabla_{\theta} \KL(p_{\bar{\pi}_{\tau,\tau^{\prime}}^*, f}  \| p_{\pi_\theta, f} ) \approx -\sum\limits_{k=1}^K{ \frac{ \exp\left\{ \omega_k \right\} }{ \sum_{j=1}^K{ \exp\left\{ \omega_j \right\}}} \nabla_{\theta} \log{\pi_\theta(\rho_k)} },
\end{equation}
\end{thm}
\begin{proof}
See Theorem \ref{thm:repmdgradientestimate}.
\end{proof}
