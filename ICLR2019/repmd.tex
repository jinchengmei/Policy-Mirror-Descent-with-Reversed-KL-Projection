%In this section we firstly discuss relative entropy regularization for policy based reinforcement learning and its relationship with mirror descent, paving the path for presenting our proposed Reversed Entropy Policy Mirror Descent (REPMD) algorithm. 
Our algorithm originates from the idea of TRPO where a KL divergence measure is used to stabilize the update of the learned policy. 
In this section w
To better understand its behavior in the parameter space, we further proves that such objective function can be re-formulated as a lift-and-project procedure, connecting TRPO to the method of relative entropy policy search.
Potential poor local optima of such 

\if0
\subsection{Policy Mirror Descent}
\label{sec:pmd}
We first present a basic policy optimization algorithm based on the idea of Mirror Descent (MD). MD has been widely used in the literature of online learning and constrained optimization problems \citep{nemirovskii1983problem,beck2003mirror}. In particular, given a \emph{reference policy} $\refPi$, let 
\begin{equation*}
\ENT(\pi_\theta; \tau, \refPi) = { \ep\limits_{\rho \sim \pi_\theta}{  r(\rho)  - \tau \KL(\pi_\theta \| \refPi) } }.
\end{equation*}

The MD update for policy optimization problem is 
\begin{equation}
%\label{eq:max_expected_reward_plus_relative_entropy}
\pi_{\theta_{t+1}} = \argmax\limits_{\pi_\theta \in \Pi}  \ENT(\pi_\theta; \tau, \pithetat), 
\end{equation}
which combines the expected reward \cref{max_expected_reward} with a relative entropy regularizer. Similar ideas have also been explored in \citet{peters2007reinforcement,wierstra2008episodic,peters2010relative,schulman2015trust,montgomery2016guided,nachum2017trust,haarnoja2018soft,abdolmaleki2018maximum}. The following \cref{prop:mirrordescent_projection} suggests an efficient implementation of the mirror descent algorithm.
Based on \cref{prop:mirrordescent_projection}, policy mirror descent (PMD) updates the policy $\pi_{\theta_{t+1}}$ by
\begin{equation}
\label{eq:pmd}
\begin{split}
&\argmin\limits_{\pi_\theta \in \Pi}{\KL( \pi_\theta \| \bar{\pi}_{\tau}^* )}, \\
\text{where}\ \ \bar{\pi}_{\tau}^* & =  \argmax\limits_{\pi \in \Delta}{ \ep\limits_{\rho \sim \pi}{  r(\rho)  - \tau \KL(\pi \| \pi_{\theta_t}) } }.
\end{split}
\end{equation}
\fi

\subsection{Revisiting Trust Region Policy Optimization (TRPO)}
\label{subsec:revisitTRPO}
Recall that TRPO\footnote{Here we present its regularized version, which is equivalent to constraint version that is presented in \citet{schulman2015trust}. } learns the policy by maximizing the expected reward with a relative entropy regularizer.
In particular, given a \emph{reference policy} $\refPi$ (usually the current policy), 
TRPO learns $\pi_{\theta_{t+1}} $ by 
%let 
\begin{equation}
%\ENT(\pi_\theta; \tau, \refPi) = 
\label{eq:max_expected_reward_plus_relative_entropy}
\pi_{\theta_{t+1}} = \argmax\limits_{\pi_\theta \in \Pi} { \ep\limits_{\rho \sim \pi_\theta}{  r(\rho)  - \tau \KL(\pi_\theta \| \refPi) } }.
\end{equation}

%\begin{equation}
%\label{eq:max_expected_reward_plus_relative_entropy}
%\pi_{\theta_{t+1}} = \argmax\limits_{\pi_\theta \in \Pi}  \ENT(\pi_\theta; \tau, \pithetat). 
%\end{equation}
%which combines the expected reward \cref{max_expected_reward} with a relative entropy regularizer. 
%Similar ideas have also been explored in \citet{peters2007reinforcement,wierstra2008episodic,peters2010relative,schulman2015trust,montgomery2016guided,nachum2017trust,haarnoja2018soft,abdolmaleki2018maximum}. 
A gradient method with line search is proposed to solve the above optimization problem in \citet{schulman2015trust}. 
To better understand its behavior in the parameter space, in this paper we instead reformulate the problem into the following lift-and-project procedure.
\begin{equation}
\label{eq:pmd}
\begin{split}
\text{\bf (Project Step)} \quad &\argmin\limits_{\pi_\theta \in \Pi}{\KL( \pi_\theta \| \bar{\pi}_{\tau}^* )}, \\
\text{\bf (Lift Step)}  \quad & \text{where}\ \ \bar{\pi}_{\tau}^* =  \argmax\limits_{\pi \in \Delta}{ \ep\limits_{\rho \sim \pi}{  r(\rho)  - \tau \KL(\pi \| \refPi) } }. 
\end{split}
\end{equation}
\cref{prop:mirrordescent_projection} shows that the two different formulations are in fact equivalent.
Note that although the lift step is non-convex in $\pi$, we can actually solve the problem analytically. In fact, by simple calculations, we have 
\begin{equation}
\label{eq:pitaustar}
\bar{\pi}_\tau^*(\rho) =  \frac{\refPi(\rho) \exp\left\{ r(\rho) / \tau \right\}}{ \sum_{\rho^{\prime}}{\refPi(\rho^{\prime}) \exp\left\{ r(\rho^{\prime}) / \tau \right\} } }.
\end{equation}
\begin{prop}
\label{prop:mirrordescent_projection}
Given a \emph{reference policy} $\refPi$,
\[
 \argmax\limits_{\pi_\theta \in \Pi} { \ep\limits_{\rho \sim \pi_\theta}{  r(\rho)  - \tau \KL(\pi_\theta \| \refPi) } } 
 = \argmin\limits_{\pi_\theta \in \Pi}{ \KL(\pi_\theta \| \bar{\pi}_\tau^*) }.
\]
\end{prop}
\begin{remk}
	The lift-and-project procedure in \cref{eq:pmd} is not completely new in the literature. In fact, it has also been proposed in ******. We postpone the detailed discussion to Section ****.
\end{remk}
\begin{remk}
The lift-and-project reformulation suggests an alternative way in solving \cref{eq:max_expected_reward_plus_relative_entropy}: Lift the current policy $\pi_{\theta_t}$ to $\bar{\pi}_\tau^*$, then perform multiple steps of gradient descent on the project step to update $\pi_{\theta_{t+1}}$. \footnote{To estimate this gradient, one would need to use the self-normalized importance sampling \cite{owen2013monte}. We omit the implementation details since it is not the main algorithm of the paper.}
Note that vanilla gradient descent methods for TRPO can be interpreted as performing only one step gradient descent for the project step. \todor[]{Is it correct?} 
\end{remk}
One can show that the above lift-and-project procedure asymptotically converges to the optimal policy when $\Pi$ is a convex set \citep{nemirovskii1983problem,beck2003mirror}. 
However, in practice, the policy $\pi_\theta$ is often parameterized by a complex non-convex function, such as a neural network, which violates the convex constraint set assumption. 
The next proposition shows that despite of the non-convexity of $\Pi$, TRPO still has some desirable properties.
\begin{prop}
\label{prop:monoto_policymirrordescent}
Given the projection step $\min\limits_{\pi_\theta \in \Pi}{\KL( \pi_\theta \| \bar{\pi}_{\tau}^* )}$ can be solved optimally, for arbitrary parametrization of $\pi$, TRPO satisfies the following properties.
\begin{enumerate}
	\item {\bf (Monotonic Improvement Guarantee)} 
	%The sequence of policies learned by TRPO is guaranteed to be monotonically improved:
	Assume that $\pi_{\theta_{t}}$ is the update sequence, then 
	 \begin{equation*}
	\oep_{\rho \sim \pi_{\theta_{t+1}}}{r(\rho)} - \oep_{\rho \sim \pi_{\theta_{t}}}{  r(\rho)} \ge 0.
	\end{equation*}
	\item {\bf (Global optimum inclusion)} Every stationary point of the expected reward $\oep_{\rho \sim \pi_\theta}{  r(\rho)}$, including the globally optimal policy $\pi_{\theta^*}$,  is a fixed point of TRPO.
	%The globally optimal policy $\pi^*$ in parameter space is included in the set of fixed points of TRPO.
\end{enumerate}
\end{prop}

\begin{remk}
	The monotonic improvement guarantee has been derived in \citep{schulman2015trust}. We give a simpler and direct proof based on our lift-and-project formulation in \cref{appsec:monoto_policymirrordescent} .
\end{remk}

Despite of its stable and reliable performance, in practice TRPO is observed to get trapped in some poor local optima. 
Indeed, while the relative entropy regularizer helps in preventing large policy update, it may also limit the exploration of TRPO.
Moreover, minimizing the KL divergence $\KL(\pi_\theta \| \bar{\pi}_\tau^*) $ is known to be \emph{mode seeking} \citep{kevin2012machine}, which can cause mode collapse during the learning process. Once the policy $\bar{\pi}_\tau^*$ drops some of the modes, learning could be trapped into sub-optimal policies.
At this point, the relative entropy regularizer will NOT encourage TRPO for further exploration.

Another problem about TRPO is that \cref{prop:monoto_policymirrordescent} relies on the condition that the projection step can be globally optimally solved.
However, oftentimes in practice this is not true when $\pi_\theta$ is non-convex in $\theta$, which hinders the applicability of \cref{prop:monoto_policymirrordescent}.

\subsection{Reversed Entropy Policy Mirror Descent \todor[]{New name}}
\label{subsec:repmd}
\if0 
First, an additional entropy regularizer, controlled by a separate parameter $\tau^{\prime}\geq 0$, is added to the objective, which encourages additional exploration as in MENT.

A similar monotonic improvement result to \cref{prop:monoto_policymirrordescent} can also be proved for REPMD but only on the $\SR(\pi_\theta) $, as shown in \cref{thm:monotonically_increasing_sr_property}, where we let
\begin{equation*}
\SR(\pi_\theta) \triangleq (\tau + \tau^{\prime})\log{ \sum_{\rho}{ \exp\left\{ \frac{r(\rho) + \tau \log{\pi_\theta(\rho)} }{\tau + \tau^{\prime}} \right\} }}
\end{equation*}
denote the softmax approximated expected reward of $\pi_\theta$.
\fi
In this section, we propose two modifications to the lift-and-project procedure to overcome its aforementioned drawbacks.
The first modification is an additional entropy regularizer to the lift step, controlled by a separate parameter $\tau^{\prime}\geq 0$, to encourage the exploration of the algorithm. 
Second, we employ the reversed \emph{mean seeking} direction of KL divergence $\KL(\bar{\pi}_{\tau,\tau^{\prime}}^* \| \pi_\theta)$ to update the policy.
The new algorithm, called ****, solves the following optimization problem to update the policy $\pi_{\theta_{t+1}}$:
\begin{equation}
\label{eq:repmd}
\begin{split}
\text{\bf (Project Step)} \quad  &\argmin\limits_{\pi_\theta \in \Pi}{\KL(\bm{ \bar{\pi}_{\tau,\tau^{\prime}}^* \| \pi_\theta }) }, \\
\text{\bf (Lift Step)} \quad  & \text{where}\ \ \bar{\pi}_{\tau,\tau^{\prime}}^*  =  \argmax\limits_{\pi \in \Delta}{ \ep\limits_{\rho \sim \pi}{  r(\rho)  - \tau \KL(\pi \| \pi_{\theta_t}) + \bm{\tau^{\prime} \cH(\pi)} }}.
\end{split}
\end{equation}
%, leading to our new algorithm.
The idea of optimizing the reverse direction of KL divergence has proven to be effective for structured prediction and reinforcement learning in previous work, such as reward augmented maximum likelihood \citep{norouzi2016reward} and UREX \citep{nachum2017improving}.
Its \emph{mean seeking} behavior would further encourage the exploration of the algorithm.
More importantly, as shown in \cref{prop:solvableprojection}, reversing the direction of the KL divergence makes the projection step solvable even for the one-layer-softmax neural network $\pi$, thus guarantees the desirable properties in practice.
\begin{prop}
	\label{prop:solvableprojection}
	Assuem $\pi_\theta(s) = \softmax(\phi_s^{\top}\theta)$. Given a reference policy $\refPi$, the projection step $\min\limits_{\theta \in \mathbb{R}^d}{\KL(\refPi \| \pi_\theta)}$ is convex in $\theta$.
\end{prop}

We now prove a similar monotonic improvement guarantee for **** on a surrogate reward $\SR(\pi_\theta) $, as shown in \cref{thm:monotonically_increasing_sr_property}.
\begin{thm}
\label{thm:monotonically_increasing_sr_property}
Assume that $\pi_{\theta_{t}}$ is the update sequence of the reversed entropy policy mirror descent algorithm, then
\begin{equation*}
	\SR(\pi_{\theta_{t+1}}) - \SR(\pithetat)\ge 0,
\end{equation*}
where
\begin{equation}
\label{eq:SR}
\SR(\pi_\theta) \triangleq (\tau + \tau^{\prime})\log{ \sum_{\rho}{ \exp\left\{ \frac{r(\rho) + \tau \log{\pi_\theta(\rho)} }{\tau + \tau^{\prime}} \right\} }}.
\end{equation}
Therefore, 
%Theorem \ref{thm:monotonically_increasing_sr_property} guarantees the monotonic improvement on $\text{SR}(\pi)$, thus implies that 
the fixed points of REPMD have a correspondence with the stationary point of $\text{SR}(\pi_\theta)$. \todor[]{Why?}

\end{thm}

\subsection{Behavior of $\SR(\pi)$}
\label{subsec:sr}
Although $\text{SR}(\pi_\theta)$ is different than the expected reward $\ep_{\rho \sim \pi_\theta}{r(\rho)}$,  in this section we present some theoretical and empirical evidences that $\SR(\pi_\theta)$ is a reasonable surrogate that may provide good guidance to the learning. 
In fact, by properly adjusting the two temperature parameters $\tau$ and $\tau^{\prime}$, $\SR(\pi_\theta)$ recovers several existing performance measures, as shown in \cref{prop:sr}.
\begin{prop}
\label{prop:sr}
$\SR(\pi_\theta)$ satisfies the following properties:
\begin{enumerate}[label=(\roman*)]
	\item  $\SR(\pi_\theta) \to \max_{\rho}{r(\rho)}$, as $\tau \to 0, \tau^{\prime} \to 0$.
	\item $\SR(\pi_\theta) \to \ep_{\rho \sim \pi_\theta}{r(\rho)}$, as $\tau \to \infty, \tau^{\prime} \to 0$. 
\end{enumerate}	
\end{prop}
\begin{remk}
	Note that $\text{SR}(\pi_\theta)$ also resembles ``softmax value function'' that appeared in value based RL \citep{nachum2017bridging,haarnoja2018soft,ding2017cold}. The standard soft value can be recovered by $\text{SR}(\pi_\theta)$ as a special case when $\tau = 0$ or $\tau'=0$. \todor[]{But \cref{prop:sr} tells that $SR(\pi)$ converges to $max_\rho$ when $\tau$ and $\tau^\prime$ are 0?}
\end{remk}

According to \cref{prop:sr}, one should gradually decrease $\tau^{\prime}$ to reduce the level of exploration as sufficient reward landscape information has been collected during the learning process. Now we can make different choices for $\tau$, depending on the policy constraint set $\Pi$.

\begin{wrapfigure}{r}{0.56\textwidth}
	\begin{minipage}{0.56\textwidth}
		\begin{algorithm}[H]
			\caption{\label{alg:repmd}  The REPMD algorithm}
			\begin{algorithmic}[1]
				\INPUT $\tau, \tau', K$
				\OUTPUT  Policy $\pi_\theta$
				\STATE Random initialized $\pi_{\theta_1}$;
				\FOR { $t=1,2,\ldots, T$ }
				\STATE Set $\refPi = \pithetat$;
				\REPEAT 
				\STATE Sample a mini-batch of $K$ trajectories from $\refPi$;
				\STATE Compute the gradient according to \cref{eq:gradient_estimator};
				\STATE Update $\pi_{\theta_{t+1}}$ by the gradient;
				\UNTIL converged or reach max\_iter;
				\ENDFOR
				\STATE Return $\pi_{\theta_T}$.
			\end{algorithmic}
		\end{algorithm}
	\end{minipage}
\end{wrapfigure}

%\begin{remk}
%\label{small_tau_choices}
Given $\tau^{\prime} \to 0$ and the reward landscape has been sufficiently explored,
the constructed unconstrained policy $\bar{\pi}_{\tau,\tau^{\prime}}^* \to \pi^*$ as $\tau \to 0$, where $\pi^*$ is the global deterministic optimal policy. 
Therefore, in the project step $\pi_\theta$ is obtained by directly projecting $\pi^*$ into $\Pi$. When the policy constraint $\Pi$ has nice properties, such as convexity, that support good behavior of KL projection, $\pi_\theta$ may achieve good performance.
However, in practice, $\Pi$ is typically non-convex. Setting $\tau \to 0$ might not work very well, since directly projecting $\pi^*$ into $\Pi$ does not always lead to a $\pi_\theta$ with large expected reward.

On the other hand, as $\tau \to \infty$ the stationary point set of $\SR(\pi_\theta)$ will approach the stationary point set of $\sum_{\rho}{ \pi_\theta(\rho) r(\rho) }$.
There exists an ideal sequence of $\tau$ values and $\tau \to \infty$ that make $\pi_\theta$ finally converge to $\pi_\theta^* \in \argmax_{\pi_\theta}{ \sum_{\rho}{\pi_\theta(\rho) r(\rho)} }$, i.e., the optimal policy in $\Pi$ with highest expected reward, recovering the target of policy optimization \cref{max_expected_reward}. \todor[]{Why?}
The following simulation results suggest that $\SR(\pi)$ could be a good guidance for maximizing the true expected reward.
A principled way to find such an ideal sequence of $\tau$ is under investigation.

\if0
Using very small $\tau$ value, i.e., $\tau \to 0$, given $\tau^{\prime} \to 0$ and the reward landscape has been sufficiently explored, the constructed unconstrained policy $\bar{\pi}_{\tau,\tau^{\prime}}^* \to \pi^*$, where $\pi^*$ is the global deterministic optimal policy. Solving the projection $\pi_\theta \leftarrow \argmin_{\pi_\theta \in \Pi}{\KL(\bar{\pi}_{\tau,\tau^{\prime}}^* \| \pi_\theta}) \approx \argmin_{\pi_\theta \in \Pi}{\KL(\pi^* \| \pi_\theta})$, we actually obtain $\pi_\theta$ policy by directly projecting $\pi^*$ into $\Pi$. When the policy constraint $\Pi$ has additional nice properties, such as convexity, that support good behavior of KL projection, this should be a good choice.
%\end{remk}
However, in practice, $\Pi$ is typically non-convex, even when $\pi_\theta$ is parameterized by a linear function approximator, let alone more complex function approximations like a neural network. In such cases, the small $\tau$ value suggested in Remark \ref{small_tau_choices} might not work very well, since directly projecting $\pi^*$ into $\Pi$ might not always lead to a $\pi_\theta$ with large expected reward.

\begin{remk}
\label{large_tau_choices}
	Given $\tau^{\prime} \to 0$ and a sufficiently explored reward landscape, as $\tau \to \infty$, the stationary point set of $\SR(\pi_\theta)$ will approach the stationary point set of $\sum_{\rho}{ \pi_\theta(\rho) r(\rho) }$.
\end{remk}
 
Remark \ref{large_tau_choices} indicates that during learning, $\pi_\theta$ may get stuck around some local maximum of $\SR(\pi_\theta)$, and increasing the $\tau$ value will lead to a different contour of $\SR(\pi_\theta)$, which might help $\pi_\theta$ escape to alternative local maxima. There exists an ideal sequence of $\tau$ values and $\tau \to \infty$ that make $\pi_\theta$ finally converge to $\pi_\theta^* \in \argmax_{\pi_\theta}{ \sum_{\rho}{\pi_\theta(\rho) r(\rho)} }$, i.e., the optimal policy in $\Pi$ with highest expected reward, recovering the target of policy optimization \cref{max_expected_reward}. A principled way to find such an ideal sequence of $\tau$ is under investigation.
 \fi
 \todor[]{Add the simulation result for true loss and $\SR(\pi)$ with different value of $\tau$.}
 
\subsection{Implementation Details}
\label{subsec:implementation}
We now discuss the implementation details of *****. The full algorithm is presented in \cref{alg:repmd}.
Similarly by simple calculation, we have the analytical solution for the lift step of ****:
\begin{equation*}
\bar{\pi}_{\tau,\tau^{\prime}}^*(\rho) \triangleq \frac{\refPi(\rho) \exp\left\{ \frac{r(\rho)-\tau^{\prime} \log \refPi(\rho) }{ \tau+\tau^{\prime}} \right\}}{ \sum_{\rho^{\prime}}{\refPi(\rho^{\prime}) \exp\left\{ \frac{r(\rho^{\prime})-\tau^{\prime} \log \refPi(\rho^{\prime})}{ \tau+\tau^{\prime}} \right\} } }.
\end{equation*}
%First note that the \emph{unconstrained optimal policy} $\bar{\pi}_{\tau,\tau^{\prime}}^*$ in (\ref{eq:repmd}) has a closed form expression.
The project step in \cref{eq:repmd}, $\min\limits_{\pi_\theta \in \Pi}{\KL(\bm{ \bar{\pi}_{\tau,\tau^{\prime}}^* \| \pi_\theta }) }$, can be optimized via stochastic gradient descent, given that one can sample trajectories from $\bar{\pi}_{\tau,\tau^{\prime}}^*$ to estimate its gradient.
To see that, note that 
\begin{equation*}
\argmin\limits_{\pi_\theta \in \Pi}\KL(\bar{\pi}_{\tau,\tau^{\prime}}^* \| \pi_\theta) = \argmin\limits_{\pi_\theta \in \Pi} \ep_{\rho \sim \bar{\pi}_{\tau,\tau^{\prime}}^* }  - \log \pi_\theta(\rho).
\end{equation*}
The next theorem shows that sampling from $\bar{\pi}_{\tau,\tau^{\prime}}^*$ can be done using self-normalized importance sampling \citep{owen2013monte}, following the idea of UREX \citep{nachum2017improving}.
\begin{thm}
\label{thm:repmdgradientestimate}
Let $\omega_k = \frac{r(\rho_k) - \tau^{\prime} \log{\refPi(\rho_k)} }{\tau + \tau^{\prime}}$. Given $K$ \emph{i.i.d.} samples $\{\rho_1, \dots, \rho_K\}$ from the \emph{reference policy} $\refPi$, we have the following unbiased gradient estimator,
\begin{equation}
\label{eq:gradient_estimator}
	\nabla_{\theta} \KL(\bar{\pi}_{\tau,\tau^{\prime}}^* \| \pi_\theta)\approx -\sum\limits_{k=1}^K{ \frac{ \exp\left\{ \omega_k \right\} }{ \sum_{j=1}^K{ \exp\left\{ \omega_j \right\}}} \nabla_{\theta} \log{\pi_\theta(\rho_k)} },
\end{equation}
%	where 
%	\[
%	\omega_k = \frac{r(\rho_k) - \tau^{\prime} \log{\refPi(\rho_k)} }{\tau + \tau^{\prime}}.
%	\] 
\end{thm}

\subsection{Cooperate with Value Function}
