
Model-free deep reinforcement learning (RL) has recently
been shown to be remarkably effective in solving
challenging sequential decision making problems
\citep{schulman2015trust,mnih2015human,silver2016mastering}.
A central method of deep RL is \emph{policy optimization}
(aka\ policy gradient),
which is based on formulating the problem
as the optimization of a stochastic objective (expected return)
with respect to the underlying policy parameterization
\citep{williams1991function,williams1992simple,sutton1998reinforcement}.
Unlike standard optimization,
stochastic optimization requires the objective and gradient to be 
\emph{estimated} from data,
typically gathered from a process depending on current parameters, 
simultaneously with parameter updates.
Such an interaction between estimation and updating
complicates the optimization process,
and often necessitates the introduction of variance reduction methods,
leading to algorithms that exhibit subtle hyperparameter sensitivity.
Joint estimation and parameter updating can also create poor local optima
whenever sampling neglect of some region
can lead to further entrenchment of a current solution.
For example, a non-exploring policy might fail to sample from high
reward trajectories,
preventing any further improvement since no useful signal is observed.
In practice, it is well known that successful application of deep RL techniques
requires a combination of extensive hyperparameter tuning,
and a large, if not impractical, number of sampled trajectories.
It remains a major challenge to develop methods that can reliably
perform policy improvement while maintaining sufficient exploration
and avoiding poor local optima, yet do so quickly.

Several ideas for improving policy optimization have been proposed
in the literature, 
generally focusing on the goals of improving stability and data efficiency
\citep{peters2010relative,van2015learning,fox2015taming,schulman2015trust,montgomery2016guided,nachum2017bridging,nachum2017trust,tangkaratt2017guide,abdolmaleki2018maximum,haarnoja2018soft}. 
Unfortunately, a notable gap remains between empirically successful methods 
and their underlying theoretical support.
Current analyses typically assume a simplified setting that either ignores the 
policy parametrization or only considers linear models.
These assumptions are hard to justify when current practice relies on 
complex function approximators, such as deep neural networks,
that are highly nonlinear in their underlying parameters.
This gulf between theory and practice is
a barrier to wider adoption of model-free policy gradient methods.

In this paper, we focus on a setting where the policy can be
parametrized as a \emph{non-convex} function of its parameters.
We begin by considering the entropy-regularized expected reward objective,
which has recently re-emerged as a foundation for state-of-the-art RL methods
\citep{williams1991function,fox2015taming,schulman2017equivalence,nachum2017bridging,haarnoja2017reinforcement}. 
Our first contribution is to reformulate the maximization of 
this objective as a lift-and-project procedure,
following the lines of Mirror Descent
\citep{nemirovskii1983problem,beck2003mirror}.
Such a reformulation achieves two things.
First, this approach makes it easier to analyze policy optimization
in the parameter space:
using this reformulation we can establish a monotonic improvement guarantee
with a fairly simple proof,
even in a non-convex setting.
We also provide a study of the fixed point properties of this setup.
Second, the proposed reformulation has practical algorithmic consequences,
suggesting, for example,
that multiple gradient descent updates should be performed
in the projection step.
These considerations lead to our first practical algorithm,
Policy Mirror Descent (PMD),
which first lifts the policy to the entire policy-simplex,
ignoring the constraint induced by its parametrization,
then approximately solves the projection step by multiple
gradient descent updates to the policy in the parameter space. 
%
% TODO: We should conclude something about this algorithm here
%

We then investigate
additional modifications to mitigate the potential deficiencies of PMD.
The main algorithm we propose, Reversed Entropy PMD (REPMD),
incorporates both an entropy and relative entropy regularizer,
and uses the mean seeking direction of KL divergence for projection.
The benefit of this approach is twofold.
First, 
using just the mean seeking direction of KL divergence for the projection step
helps avoids poor local optima;
second,
this specific problem can now be efficiently solved to global optimality
in certain non-trivial \emph{non-convex} scenarios,
such as one-layer-softmax networks.
Such a result has not been previously reported 
in the policy optimization literature.
%
% TODO: make sure!
%
Similar guarantees can be proved for REPMD,
which additionally incorporates entropy regularization,
with respect to a surrogate objective $\SR(\pi)$.
We further study the properties of $\SR(\pi)$ and provide theoretical
and empirical evidence that $\SR$ 
can effectively guide the search for good policies.
Finally, we also show how this algorithm can be extended 
with a value function approximator,
and develop an actor-critic version that is effective in practice.



\subsection{Notations and Problem Setting}
\label{subsec:notations_and_settings}

For simplicity, we only consider finite horizon
settings with finite state and action spaces. 
The behavior of an agent is modelled by a policy $\pi(a|s)$
that specifies a probability distribution over a finite set of actions
given an observed state. 
At each time step $t$, the agent takes an action $a_t$ by sampling from
$\pi(a_t | s_t)$.
The environment then returns a reward $r_t = r(s_t, a_t)$ and the next state
$s_{t+1} = f(s_t, a_t)$,
where $f$ is the transition function not revealed to the agent.
Given a trajectory, a sequence of states and actions
$\rho=(s_1, a_1, \dots, a_{T-1}, s_T)$,
the policy probability and the total reward of $\rho$ are defined as
$\pi(\rho) = \prod_{t=1}^{T-1} \pi(a_t| s_t)$
 and $r(\rho) = \sum_{t=1}^{T-1} r(s_t, a_t)$. 
Given a set of parametrized policy functions $\pi_\theta \in \Pi$,
policy optimization aims to find the optimal policy $\pi_\theta^*$
by maximizing the expected reward,
\begin{equation}
\label{max_expected_reward}
\begin{split}
\pi_\theta^* \in \argmax_{\pi_\theta \in \Pi}{ \ep\limits_{\rho \sim \pi_\theta}{r(\rho)} },
\end{split}
\end{equation}

We use
$\Delta \triangleq \{ \pi | \sum_{\rho}{\pi(\rho)} = 1, \pi(\rho) \ge 0,
\forall \rho \}$
to refer to the probability simplex over all possible trajectories. 
Without loss of generality, we also assume that the state transition function
is deterministic, and the discount factor $\gamma = 1$.
This same simplification is also assumed in \citet{nachum2017improving}. 
Results for the case of a stochastic state transition function are presented
in \cref{sec:stochasticsetting}.
