Model-free deep reinforcement learning (RL) has recently been demonstrated to successfully solve a wide range of difficult sequential decision making problems \citep{schulman2015trust,mnih2015human,silver2016mastering}, it also significantly introduces additional complications in understanding the behavior of the model. 

In practice, policy based Deep RL is different than a traditional optimization problem, in the sense that the argument to be optimized, i.e. the policy, is also used to collect training data from the environment. This interaction may lead to lack of exploration, since the learner's policy may get stuck in a local optimum and fail to collect high reward trajectories, preventing any learning from useful signals.  An on-line learning algorithm should have the ability to explore the policy space properly and efficiently, to avoid getting trapped in a locally optima, while discovering the globally optimal policy quickly.

In this paper, we propose a method with a better exploration strategy, such that (a) it retains the exploration efficiency of existing methods; (b) it monotonically increases its chance of exploring trajectories generated by the optimal policies, or evolving closer to the optimal policies. Our proposed method Reversed Entropy Policy Mirror Descent (REPMD), takes both the entropy and relative entropy regularizers. Unlike common policy gradient based methods, REPMD is in a two-stage manner.  REPMD first updates the policy in the entire policy-simplex, ignoring the constraint induced by its parametrization, then a projection step is performed to update the policy in the parametrized policy space. Such a two-stage update guarantees REPMD to increase performance monotonically. The proposed REPMD method is then justified from both theoretical and empirical perspectives.

The rest of the paper is organized as follows. After introducing exploration of RL in \cref{sec:exploration_in_policy_optimization}, we propose the REPMD method in \cref{sec:reversed_emtropy_policy_mirror_descent}. We provide the analysis for the monotonically increasing performance property of REPMD in \cref{sec:reversed_emtropy_policy_mirror_descent}, and conduct experiments to validate our algorithm in \cref{sec:experiments}. Some related work is discussed in \cref{sec:related_work}, and the conclusion and directions for future work are presented in \cref{sec:conclusion_and_future_work}.

The lift-and-project formulation differs our method from most of the policy search methods. 
On the one hand, such formulation provides an easier way in analyzing the behavior of the algorithm in parameter space. For example, the monotonical improvement guarantee can be proved in a fairly direct and simple way in this paper, even for some particular non-convex $\pi_\theta$. 
On the other hand, such formulation suggests to perform multiple steps of gradient descent on the project step with the current policy fixed as the reference policy. As shown in \cref{sec:experiments}, multiple steps of gradient descent leads to a significant improvement in performances compared to single step of gradient descent.

%The benefit of this, as discussed in \cref{subsec:repmd} has two folds: (1) Using the mean seeking KL divergence in the project step helps avoids some poor local optima; (2) The problem can still be globally optimally solved for one-layer networks which is non-convex.
We assume discount factor to be 1. Assume deterministic state transition function (WLOG). 

\subsection{Notations and Settings}
\label{subsec:notations_and_settings}

We consider finite horizon reinforcement learning settings with finite state and action spaces. The behavior of an agent is modelled by a policy $\pi(a|s)$, which estimates a probability distribution over a finite set of actions given an observed state. At each time step $t$, the agent takes an action $a_t$ by sampling from $\pi(a_t | s_t)$. The environment then returns a reward $r_t = r(s_t, a_t)$ and the next state $s_{t+1} = f(s_t, a_t)$, where $f$ is the transition function and it is not revealed to the agent. Given a trajectory, a sequence of states and actions $\rho=(s_1, a_1, \dots, a_{T-1}, s_T)$, the policy probability and the total reward of $\rho$ are defined as $\pi(\rho) = \prod_{t=1}^{T-1} \pi(a_t| s_t)$ and $r(\rho) = \sum_{t=1}^{T-1} r(s_t, a_t)$. We use $\Delta \triangleq \{ \pi | \sum_{\rho}{\pi(\rho)} = 1, \pi(\rho) \ge 0, \forall \rho \}$ to refer to the probabilistic simplex over all possible trajectories. For simplicity we also assume that the state transition function is deterministic, while our results can be easily extended to the general setting with random transition functions.